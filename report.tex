\documentclass{report}

%% Language and font encodings
\usepackage[english]{babel}
\usepackage[utf8x]{inputenc}
\usepackage[T1]{fontenc}

%% Sets page size and margins
\usepackage[top=3cm,bottom=2cm,left=3cm,right=3cm,marginparwidth=1.75cm]{geometry}

%% Useful packages
\usepackage{amsmath}
\usepackage{graphicx}
\usepackage[colorinlistoftodos]{todonotes}
\usepackage[colorlinks=true, allcolors=black]{hyperref}

% Specify bibliography package
\usepackage{natbib}



\title{Review of Assigned Readings}
\author{Changhwan(Peter) Yoo}
\date{LING/TRST 415: Spring 2018}



\begin{document}
\renewcommand{\chaptername}{Day}
\maketitle
\tableofcontents

% For each day of class, you will have a new chapter
\chapter{Tuesday 30 January 2018: Machine Translation in the 1940s".}
% You should have one section per assigned reading
\section{Milestones in MT - How it all began in 1947 and 1948}
 The idea about Machine Translation came about naturally during the World War II, as the increasing interests toward computer which was referred to as ‘Electric Brains’(Hutchins 22) arouse on the surface together with number crunching. However, it only was introduced to a few number of people, and one of them was Warren Weaver, who worked at Rockefeller Foundation. He was assigned to develop various military weapons because of his previous career as a professor at University of Wisconsin and his position being responsible for this role allowed him to meet many renowned scientists. Later, he realized and felt a strong need for distributing scientific knowledge in many areas through programming such as Agricultural Production in Mexico along the way and also understanding internationally for related areas. In the midst of all these, he thought that he should have translating machine although he was aware of its limitation in having a thorough version of translation insisting of scientific intellectual translation rather than a well-formed sentences. However, Weiner strongly disagreed and responded that such translated version requires the same effort in translating from the source text because it is too ambiguous. There was one time D.Booth from England came over to the U.S to see electronic computer newly invented, and also visited Rockefeller Foundation to give thanks for the fund. In Booth’s visit to Rockefeller, he also talked to Weaver for further fund, and Weaver suggested inventing on a simple translation machine, promising the possibility of funds provision. However, they met together a few years later when Weaver knew that Richens’ simple word for word translation method and Booth work together to come up with a new one. Booth explained that his ideas had a limitation. Weaver thought that an appropriate concept of understanding translation machine needs to first be explained.
 
% Within each section, write a summary of that reading.
%
% Use \citet to cite a source inline
% Use \citep to cite a source parenthetically

\section{Weaver and Weiner, Correspondense}
 Warren begins his letter by his anxiety towards the limitation of communication between peoples due to lack of means. And suggests a new invention of translating machine just to translate for scientific materials word for word, not asking for some beautiful sentences. Also, he reminds Nobert of the cryptography being a fascinating method in translation. Nobert writes to Warren saying that he thought that the method of translation Warren is suggesting is not practical as there are words such as ‘get’ which has numerous meaning that is hard to differentiate and concludes that machine translation now is hard to go beyond the ideas that the blinds use. However, Warren sends him a letter back by showing some disappointment on Nobert’s underestimation towards his ideas. He maintains his idea by demonstrating on how such word as ‘get’ needs to be put into the computer using two-word combination as one word.

\section{Weaver and Prager, Correspondence}
Warren sends an email to Prager to ask if what he told him long ago regarding coded message used in various cryptographic ways showing his interests toward it, but rather carefully due to the fact that Warren thought it might be confidential in some ways to military uses. And goes on to question what he supposes a finding could be found from the mathematician’s method. Warren elaborates that there might be a quality for a language that is compatible for all languages. And he thinks that it could link to a solution of translation from one language to another which they long had a problem with. Then, Warren gets his email back saying that what he remembers is right and Prager also permits Warren to utilize the method by telling him how to use it, which was simple by a change of similar letters in Turkish to English. Prager uses this chance to inform him of the changes that made in the University according to their need. In the response to the letter Warren was glad to see the appointments that had been made and another letter sent to Warren corresponding to his question on the number(80 to 100) of letters used in the method they were talking about.

\section{Milestones in MT - Warren Weaver's 1949 memorandum}
This article is about Weaver’s memorandum which took a significant role in research and development of machine translation. It begins with his capability of being responsible for the job based on his previous accomplishment. This memorandum was basically proposed for a further fruitful outcome due to the limitation of word for word translation in the previous experiments. In developing a new translation, he first confronted with a problem with words that have multiple meanings. And solution to this was to extend a sentence lengthy enough to find a right word through the context. Secondly, He supposed there are logical elements in languages and believed that translation is possible through a computer with function of repeating regulations of certain language character by tracing the words according to finite set of premises(Hutchins 22). As the third proposal, he was once attracted by a cryptographic method involving the entire cryptographic field but had to leave it unsolved due to inaccessibility. He concludes by being very optimistic about universal language by saying that translation may not be the matter of word for word translation rather finding a base structure of language that can be applied for all languages.

\section{Translation}
The memorandum contains his everlasting hope for inventing a translation using computer but it is very clear that he is doing so for the sake of easy international communication, which he believes will lead to a huge development in various areas. In the story, there was a trial with Turkish which was encoded, but wasn’t successful. However, one interesting fact was that when the unsuccessful encoding letters were properly put in order, someone who never knew that it was Turkish actually interpreted the coding. And he relates this to how Japanese did to decode during World War I, emphasizing that once a person knows how the coded letters work, one can understand the text regardless of what language it was encoded from. Then he comes up with an idea that although it might be very minimal there’s a common element for all languages to a certain extent. He also offers a scientific facts that regardless of language kind, one use the same part of the brain in processing the same kind sentence, and further explains by giving an example that tree in one region is the same tree in other region or country. And in later pages, his thoughts in previous articles on usefulness of MT, but not for literary works, the limitation of cryptographic way of MT, how to deal with it, and so on is repeated.
\bibliographystyle{apalike}
\bibliography{references}

Hutchins, Milestones in MT - How it all began in 1947 and 1948

Weaver and Wiener, Correspondence

Weaver and Prager, Correspondence

Hutchins, Milestones in MT - Warren Weaver's 1949 memorandum

Weaver, Translation

\end{document}